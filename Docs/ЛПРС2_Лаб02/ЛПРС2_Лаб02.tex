
\documentclass[12pt]{article}

%%%%%%%%%%%%%%%%%%%%%%%%%%%%%%%%%%%%%%%%%%%%%%%%%%%%%%%%%%%%%%%%%%%%%%%%%%%%%%
% Packages and settings.

\usepackage{geometry}
\geometry{
	a4paper,
	%total={210mm,297mm},
	left=20mm,
	right=20mm,
	top=20mm,
	bottom=20mm,
}

%% Every section new page.
%\usepackage{titlesec}
%\newcommand{\sectionbreak}{\clearpage}

% Must be before babel.
\usepackage{siunitx}
\usepackage[toc,nonumberlist,acronym]{glossaries}

\usepackage[T1,T2A]{fontenc}
\usepackage[utf8]{inputenc}
\usepackage[serbianc,serbian,english]{babel}

%\bibliographystyle{IEEEtran} % Because biblatex do not work with babel.
%\usepackage[nottoc,numbib]{tocbibind} % To have Literature in TOC.

\usepackage{cite}

\usepackage[breaklinks]{hyperref}
%\usepackage{cleveref} % Hard to work with serbian.

\usepackage{indentfirst}

\usepackage{amsmath}
\usepackage{amsfonts}

\usepackage{tikz}

% To force image position.
\usepackage{float}


\usepackage{array}
\usepackage{makecell}
\usepackage{multirow}

%%%%%%%%%%%%%%%%%%%%%%%%%%%%%%%%%%%%%%%%%%%%%%%%%%%%%%%%%%%%%%%%%%%%%%%%%%%%%%
% Correction of babel.

\addto\captionsserbianc{%
	\renewcommand{\listfigurename}{Листа слика}%
	\renewcommand{\listtablename}{Листа табела}%
}

\addto\captionsserbian{%
	\renewcommand{\listfigurename}{Lista slika}%
	\renewcommand{\listtablename}{Lista tabela}%
}

%%%%%%%%%%%%%%%%%%%%%%%%%%%%%%%%%%%%%%%%%%%%%%%%%%%%%%%%%%%%%%%%%%%%%%%%%%%%%%
% Commands.

\def\regWD{-4mm}
\def\regHO{-30mm}
\def\regH{-13mm}
\def\regRWO{\regH-0mm}
\def\regIVO{\regRWO-5mm}
\newcommand{\regBitLegend}[1]{%
	\node[anchor=south east] at (32*\regWD, #1*\regHO) {Bits};
}
\newcommand{\regBit}[2]{%
	\node[anchor=south] at (#2*\regWD+0.5*\regWD, #1*\regHO) {#2};
}
\newcommand{\regAddr}[2]{%
	%\node[anchor=south east] at (32*\regWD, #1*\regHO+0.5*\regH) {reg idx (rel/abs)};
	%\node[anchor=north east] at (32*\regWD, #1*\regHO+0.5*\regH) {#2};
	\node[anchor=east] at (32*\regWD, #1*\regHO+0.5*\regH) {#2};
}
\newcommand{\regField}[4]{%
	\draw (#2*\regWD+\regWD, #1*\regHO) rectangle (#3*\regWD, #1*\regHO+\regH)
		node[pos=.5] {#4};
}
\newcommand{\regFieldThin}[4]{%
	\draw (#2*\regWD, #1*\regHO) rectangle (#3*\regWD+\regWD, #1*\regHO+\regH)
		node[pos=.5, rotate=90] {#4};
}
\newcommand{\regRWLegend}[1]{%
	\node[anchor=north east] at (32*\regWD, #1*\regHO+\regRWO) {Read/Write};
}
\newcommand{\regRW}[4]{%
	\node[anchor=north] at 
		(
			#2*0.5*\regWD + #3*0.5*\regWD + 0.5*\regWD,
			#1*\regHO+\regRWO
		)
		{#4};
}
\newcommand{\regInitLegend}[1]{%
	\node[anchor=north east] at (32*\regWD, #1*\regHO+\regIVO) {Intial Value};
}
\newcommand{\regInit}[4]{%
	\node[anchor=north] at 
		(
			#2*0.5*\regWD + #3*0.5*\regWD + 0.5*\regWD,
			#1*\regHO+\regIVO
		)
		{#4};
}
\newcommand{\regStd}[2]{%
	\regBitLegend{#1}
	\regAddr{#1}{#2}
	\regRWLegend{#1}
	\regInitLegend{#1}
	\regBit{#1}{31}
	\regBit{#1}{0}
}


%%%%%%%%%%%%%%%%%%%%%%%%%%%%%%%%%%%%%%%%%%%%%%%%%%%%%%%%%%%%%%%%%%%%%%%%%%%%%%
% Title.

\title{
	ЛПРС2 Лаб 02\\
	{\normalsize Комплексно Меморијско Мапирање}\\
	{\small верзија 2.2}
}
\author{Милош Суботић}

\date{\today}

%%%%%%%%%%%%%%%%%%%%%%%%%%%%%%%%%%%%%%%%%%%%%%%%%%%%%%%%%%%%%%%%%%%%%%%%%%%%%%

\begin{document}

\sloppy % For text not going over margings.

% Babel language of built-in standardized sections, like Table of Content.
\selectlanguage{serbianc}
%\selectlanguage{serbian}

%%%%%%%%%%%%%%%%%%%%%%%%%%%%%%%%%%%%%%%%%%%%%%%%%%%%%%%%%%%%%%%%%%%%%%%%%%%%%%
% Starting stuff.

\clearpage
\maketitle
% No page number on title page.
\thispagestyle{empty}
%\pagebreak

%%%%%%%%%%%%%%%%%%%%%%%%%%%%%%%%%%%%%%%%%%%%%%%%%%%%%%%%%%%%%%%%%%%%%%%%%%%%%

\section{SW\_PIO и LED\_PIO}
\par
У претходној вежби смо користили \verb|SW_PIO| и \verb|LED_PIO|
периферијске IP-јеве.
Њихова меморијска мапа регистара је дата на
Слици~\ref{fig:sw_pio_mm} и Слици~\ref{fig:led_pio_mm}.

\begin{figure}[H]
	\centering
	\begin{tikzpicture}
		\regStd{0}{0 (0x00/0x9000)}
		\regBit{0}{8}
		\regBit{0}{7}
		\regField{0}{31}{8}{-}
		\regRW{0}{31}{8}{R}
		\regInit{0}{31}{8}{0}
		\regField{0}{7}{0}{SW[7:0]}
		\regRW{0}{7}{0}{R}
		\regInit{0}{7}{0}{0x00}
	\end{tikzpicture}
	
	\caption{\textsc{SW\_PIO} меморијска мапа}
	\label{fig:sw_pio_mm}
\end{figure}

\begin{figure}[H]
	\centering
	\begin{tikzpicture}
		\regStd{0}{0 (0x00/0x9010)}
		\regBit{0}{8}
		\regBit{0}{7}
		\regField{0}{31}{8}{-}
		\regRW{0}{31}{8}{R}
		\regInit{0}{31}{8}{0}
		\regField{0}{7}{0}{LED[7:0]}
		\regRW{0}{7}{0}{RW}
		\regInit{0}{7}{0}{0x05}
	\end{tikzpicture}
	
	\caption{\textsc{LED\_PIO} меморијска мапа}
	\label{fig:led_pio_mm}
\end{figure}

\par
Са Слике~\ref{fig:sw_pio_mm} се може видети да
\verb|SW_PIO| има само 1 регистар.
Читајући "0 (0x00/0x9000)",
може се закључити да је то 0-ти регистар,
са \textbf{релативном адресом} 0x00 (релативном унутар тог IP-ја)
и \textbf{апсолутном адресом} 0x9000.
Само доњих 8 бита ([7:0]) је мапирано на бите прекидача,
док су остали виши бити ([31:8]) \textbf{резервисани} (назначено са -).
Овај регистар дозвољава само читање и он је такозвани \textbf{Read-Only}.
Резервисани бити се могу само читати, док се писање у њих игнорише.
Исто важи и за SW[7:0].
Писање у тај регистар се одбацује,
док се могу само прочитати вредности прекидача.
Почетна вредност резервисаних бита је 0 и то ће бити резултат читање тих бита.
Почетна вредност за SW[7:0] није од великог значаја,
јер након пуштања ресета, ти бити ће имати вредност прекидача.
\par
Сличне важи и за \verb|LED_PIO| на Слици~\ref{fig:led_pio_mm}.
Разлика је у томе што LED[7:0] може да се чита и пише
и он је такозвани \textbf{Read-Write} регистар.
Приликом читања овог регистра,
прочитаће се оно што је у њега претходно записано
тј. његова почетвна вредност ако ништа до тада није било писано у њега.
Његова почетна вредност је 0x05.
\par
Игнорисање уписа може бити погодно,
јер не морамо водити рачуна да ли су виши бити постављени на 0.
Ако рецимо у коду имамо у 32-битној променљивој 0xbabadeda
и желимо уписати у LED[7:0] 0xda,
довољно је да на адресу 0x9010 упишемо у 32-битни регистар
целу 32-битну променљиву са 0xbabadeda,
без да морамо да маскирамо горњих 24 бита (\& са 0x000000ff).
\par
SW[7:0] регистар захтева volatile декоратор приликом приступа,
јер компајлер нити софтвер нису у могућности да знају
коју вредност може да има овај регистар,
јер хардвер је може променити независно од софтвера.
С друге стране хардвер неће мењати LED[7:0],
и софтвер може знати његову вредност,
па изгледа да је могуће изоставити volatile декоратор.
Међутим, компајлер може приметити да
иако се пише у ту меморијску локацију, никад се не чита из исте.
Компајлер ће онда извршити оптимизацију тог писања,
тј. неће ни писати у тај регистар,
и притом нам исписати упозорење да
пишемо у промениљиву али никада је не читамо.

\section{SW\_LED\_PIO}
\par
У овој вежби ћемо користити нову прериферију, под називом \verb|SW_LED_PIO|.
Ова нова периферија у себи садржи функционалност
\verb|SW_PIO| и \verb|LED_PIO| периферија,
уз неке нове додатне функционалности.
Њена меморијска мапа
Слици~\ref{fig:sw_led_pio_mm_1} и Слици~\ref{fig:sw_led_pio_mm_2}.

\begin{figure}[H]
	\centering
	
	\newcommand{\regUnpackSwLed}[3]{%
		\regStd{#1}{#1 (0x#2/0xa0#2)}
		\regField{#1}{31}{1}{-}
		\regRW{#1}{31}{1}{R}
		\regInit{#1}{31}{1}{0}
		\regFieldThin{#1}{0}{0}{{\tiny SW#1/LED#1}}
		\regRW{#1}{0}{0}{RW}
		\regInit{#1}{0}{0}{#3}
	}
	
	\begin{tikzpicture}
		\regUnpackSwLed{0}{00}{0/1}
		\regUnpackSwLed{1}{04}{0/0}
		\regUnpackSwLed{2}{08}{0/1}
		\regUnpackSwLed{3}{0c}{0/0}
		\regUnpackSwLed{4}{10}{0/0}
		\regUnpackSwLed{5}{14}{0/0}
		\regUnpackSwLed{6}{18}{0/0}
		\regUnpackSwLed{7}{1c}{0/0}
	\end{tikzpicture}
	
	\caption{\textsc{SW\_LED\_PIO} меморијска мапа 1/2}
	\label{fig:sw_led_pio_mm_1}
\end{figure}

\begin{figure}[H]
	\centering
	
	\newcommand{\regRes}[3]{%
		\regStd{#1}{#2 (0x#3/0xa0#3)}
		\regField{#1}{31}{0}{-}
		\regRW{#1}{31}{0}{R}
		\regInit{#1}{31}{0}{0xbabadeda}
	}
	
	\begin{tikzpicture}
		\regStd{0}{8 (0x20/0xa020)}
		\regBit{0}{8}
		\regBit{0}{7}
		\regField{0}{31}{8}{-}
		\regRW{0}{31}{8}{R}
		\regInit{0}{31}{8}{0}
		\regField{0}{7}{0}{SW[7:0]/LED[7:0]}
		\regRW{0}{7}{0}{RW}
		\regInit{0}{7}{0}{0x00/0x05}
		
		\regStd{1}{9 (0x24/0xa024)}
		\regBit{1}{8}
		\regBit{1}{7}
		\regField{1}{31}{8}{-}
		\regRW{1}{31}{8}{R}
		\regInit{1}{31}{8}{0}
		\regField{1}{7}{0}{\tiny SW\_CHANGED[7:0]}
		\regRW{1}{7}{0}{R}
		\regInit{1}{7}{0}{0x00}
		
		\regStd{2}{10 (0x28/0xa028)}
		\regBit{2}{1}
		\regField{2}{31}{1}{-}
		\regRW{2}{31}{1}{R}
		\regInit{2}{31}{1}{0}
		\regFieldThin{2}{0}{0}{\tiny SET\_LEDS}
		\regRW{2}{0}{0}{W}
		\regInit{2}{0}{0}{0}
		
		\regStd{3}{11 (0x2c/0xa02c)}
		\regField{3}{31}{0}{INVERT\_LEDS}
		\regRW{3}{31}{0}{W}
		\regInit{3}{31}{0}{0}
		
		\regRes{4}{12}{30}
		\regRes{5}{13}{34}
		\regRes{6}{14}{38}
		\regRes{7}{15}{3c}
	\end{tikzpicture}
	
	\caption{\textsc{SW\_LED\_PIO} меморијска мапа 2/2}
	\label{fig:sw_led_pio_mm_2}
\end{figure}
\par
Регистри са индексом од 0-тог до 7-ог
су такозвани распаковани (\textbf{Unpacked}).
Најнижи тј. 0-ти бит сваког регистар садржи
један бит из SW односно LED.
Одговарајући индекс регистра одговара одговарајућем биту.
Писањем у тај нанижи бит резултује писањем на ледаче,
док читање тог бита даје вредност прекидача.
Овакво меморијско мапирање омогућује брз приступ појединачним
битима тј. пиновима.
Довољно је само прочитати тј. уписати целобројну вредност 0 или 1,
да би се постигао жељени резултат,
без икаквог маскирања и померања.
Треба запазити да сада постоје две почетне вредности,
једна за читање и друга за писање.
\par
С друге стране, 8. регистар садржи је пакован (\textbf{Packed}).
У њему је свих 8 бита из SW[7:0] и LED[7:0],
као што је већ постојало код \verb|SW_PIO| и \verb|LED_PIO| периферија.
Разлика је у томе што док су код периферија \verb|SW_PIO| и \verb|LED_PIO|
били одвојени регистри за прекидаче и ледаче,
сада су спојени у један,
где читањем се приступа прекидачима,
а писањем ледачама.
Овако меморијско мапирање је погодно када је потребно мењати више,
а поготово све бите.
\par
Регистар 9 је SW\_CHANGED[7:0].
У овом регистру 1 индикује да је одговарајући прекидач промењен у прошлости.
Овај регистар има бочни ефекат (\textbf{Side effect}).
Бочни ефекад се огледа у томе да приликом читања овог регистра,
он се и ресетује на 0.
Након ресета, овај регистар поновно прати измене прекидача.
Уколико прекидач није измењен, приликом идућег читања,
одговарајући бит ће бити 0.
Овај регистра јер Read-Only.
\par
0-ти бит регистар 10 је SET\_LEDS.
Тај бит је \textbf{Write-Only}.
Писањем 0 у тај бит, изазваће да све ледаче буду 0,
док писањем 1, све ледаче ће бити 1.
Његово читање ће дати вредност 0.
Овај регистар у својој реализацији нема меморију,
која памти да ли је уписана 0 или 1,
већ је врста командног регистра.
Он у том тренутку изврашава операцију
на меморији тј. RTL регистру LED[7:0].
Стога читање овог регистра ни нема неког смисла,
па је из тог разлога он Write-Only.
\par
По том питању 11. регистар под називом INVERT\_LEDS,
је још интересантнији.
Приликом уписа у њега,
он ће извршити инвертовање ледача,
без обзира која се вредност уписује.
\par
Остали регистри су резервисани и њихово читање ће дати 0xbabadeda.

%TODO \section{Софтверски аутомат}

\section{Задатак}
\par
Направити штоперицу користећи \verb|SW_LED_PIO| периферију
у \verb|custom_pio_02_stopwatch| апликацији.
Леви у клупи треба да користи распаковане прекидаче и паковане ледаче,
док десни у клупи треба да користи паковане прекидаче и распаковане ледаче.




%%%%%%%%%%%%%%%%%%%%%%%%%%%%%%%%%%%%%%%%%%%%%%%%%%%%%%%%%%%%%%%%%%%%%%%%%%%%%%

%\pagebreak
%\bibliography{cites}

%%%%%%%%%%%%%%%%%%%%%%%%%%%%%%%%%%%%%%%%%%%%%%%%%%%%%%%%%%%%%%%%%%%%%%%%%%%%%%

\end{document}
